\documentclass{latex2man}

\begin{Name}{1}{shaman}{Sam Stuewe}{Fetch Weather Information}{Shaman - A lightweight tool to fetch weather information from the National Weather Service}
   \Prog{Shaman} - A lightweight tool to fetch weather information from the National Weather Service
\end{Name}

\section{Synopsis}
\Prog{shaman}
   \oOpt{-h}
   \oOptArg{-f}{\SP "format"}
   \oOpt{-i|-m}
   \oOptArg{-l}{\SP "location"}

\section{Description}
\Prog{Shaman} is a very lightweight application to fetch weather information from the National Weather Service.

\section{Options}
\begin{description}
   \item[\Opt{-h}, \Opt{--help}]
      Print usage information and exit
   \item[\OptArg{-f}{\SP "format"}, \OptArg{--format=}{"format"}]
      Specify an output format
   \item[\Opt{-i, --imperial}]
      Print information using Imperial units
   \item[\OptArg{-l}{\SP "location"}, \OptArg{--location=}{"location"}]
      Specify location to query. Note that \Arg{"location"} may be given either as a zipcode or as "City, ST"
   \item[\Opt{-m, --metric}]
      Print information using Metric units
\end{description}

\section{Format}
The following escapes may be used in the formatString:

\begin{verbatim}
%%    A literal percent sign
%c    Weather condition
%d    Relative Humidity
%D    Dew point
%h    Heat Index
%H    Hazard Warnings
%p    Pressure
%P    Probability of precipitation
%r    Reporter Identity
%R    Reporter Coordinates
%t    Temperature
%v    Visibility
%w    Wind Speed
%W    Wind Direction
\end{verbatim}
Some other basic escape characters may be used such as newline characters (\Bs n) and literal backslashes (\Bs\Bs).

\section{Author}
Copyright \copyright 2012-2013 Sam Stuewe\\
License GPLv2: GPL version 2 \URL{https://www.gnu.org/licenses/gpl-2.0.html} \\
This is free software: you are free to change and redistribute it. \\
There is NO WARRANTY, to the extent permitted by law.

Feature requests, bug reports and other comments may be submitted via GitHub (\URL{https://github.com/HalosGhost/shaman.git}) or E-mail (\Email{halosghost@archlinux.info})

\section{See Also}
\Cmd{libcurl}{3}

\LatexManEnd
